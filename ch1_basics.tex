%! Author = akalanitski
%! Date = 22.04.25

\documentclass[12pt]{article}
\usepackage[a4paper,left=2cm, right=1cm, top=1cm, bottom=2cm]{geometry}         % For setting margins

% Optional packages
\usepackage[utf8]{inputenc}   % Encoding
\usepackage[T1]{fontenc}      % Better fonts
\usepackage{multicol}
\usepackage{listings}
\usepackage{xcolor}
\usepackage[framemethod=default]{mdframed}
\usepackage[scaled]{beramono}
\usepackage{tocloft} % Required for custom Table of Contents formatting



% Make hyperlinks inside of document and clickable TOC
\usepackage[
    colorlinks=true,              % Enable colored links
    linkcolor=black,              % TOC, sections, etc.
    urlcolor=blue,                % URLs (from \href or \url)
    pdfborder={0 0 1}             % Underline style
]{hyperref}

% COLUMNS LAYOUT and STYLES
\setcounter{secnumdepth}{0}
\renewcommand{\cftsecnumwidth}{0pt} % Remove section numbers from the TOC
\setlength{\parindent}{0pt}       % First line indentv
\setlength{\parskip}{1em}         % Add vertical space between paragraphs

% COLUMNS
\usepackage{paracol}
\columnratio{0.7}
\setlength{\columnseprule}{0.4pt} % Line between columns
\setlength{\columnsep}{1em}       % Space between columns

\newmdenv[
    linecolor=gray,
    linewidth=1pt,
    roundcorner=5pt,
    backgroundcolor=gray!10,
    innertopmargin=1em,
    innerbottommargin=1em,
    innerleftmargin=1em,
    innerrightmargin=1em
]{Note}


% Go language definition for listings
\lstdefinelanguage{Go}{
    morekeywords={
        break, case, chan, const, continue, default, defer, else, fallthrough,
        for, func, go, goto, if, import, interface, map, package, range,
        return, select, struct, switch, type, var, nil, true, false, make,
        new, append, cap, copy, delete, len, close, complex, real, imag
    },
    sensitive=true,
    morecomment=[l]{//},
    morecomment=[s]{/*}{*/},
    morestring=[b]",
    morestring=[b]`,
}

% Style configuration
\lstset{
    language=Go,
    numbers=left,                % Line numbers on the left
    numbersep=10pt,
    numberstyle=\ttfamily\small\color{darkgray},
    basicstyle=\ttfamily\small,
    keywordstyle=\color{blue}\bfseries,
    commentstyle=\color{gray},
    stringstyle=\color{orange},
    rulecolor=\color{gray},
    showstringspaces=false,
    breaklines=true,
    tabsize=1,
}

\begin{document}

% First page with formal info and document structure.
% Title page
\title{Go Lang Notes}
\author{Aliaksei Kalanitski}
\date{\today}
\maketitle

My notes during the GOLang basic course. Cover only very basic topics like
Function, Variables, Input, Output, Keywords

% Table of contents
\begin{multicols}{2}
    \tableofcontents
\end{multicols}
\newpage


\section{Syntax}

\begin{multicols}{2}
    \subsection{Keywords}
    \fcolorbox{blue!05}{lightgray}{\ttfamily
        \begin{tabular}{@{\hskip 0pt}llll}
            if    & const   & break   & interface \\
            else  & case    & defer   & select\\
            for   & type    & package & continue\\
            goto  & chan    & switch  & default \\
            map   & range   & struct  & fallthrough\\
            var   & func    & import  & return \\
            go \\
        \end{tabular}
    }
    \subsection{Identifiers}
    Reserved function identifiers also called build-in functions that
    available everywhere without importing

    \fcolorbox{blue!05}{lightgray}{\ttfamily
    \begin{tabular}{@{\hskip 0pt}lllll}
            len  & copy & delete & close & complex  \\
            cap  & imag & append & print  & recover \\
            new  & make & panic  & real \\
    \end{tabular}
    }

    Standard types

    \fcolorbox{blue!05}{lightgray}{\ttfamily
    \begin{tabular}{@{\hskip 0pt}llll}
        any & comparable & error & uintptr \\
        int & uint & \ldots \\
    \end{tabular}
    }

    Constants and values
    \fcolorbox{blue!05}{lightgray}{\ttfamily
    \begin{tabular}{@{\hskip 0pt}lllll}
        true & false & iota & nil & NaN? \\
    \end{tabular}
    }

    Blank identifier
    \texttt{_}
\end{multicols}


\subsection{Operators}
\begin{paracol}{2}
\begin{leftcolumn}
\begin{tabular}{lll}
math        & + -       & P2 binary plus and minus \\
            & * /       & P1 binary multiply, divide \\
            & \%        & P1 binary reminder \\
bitwise     & \& \&\^{} & P1 Binary ``and'', ``not and'' \\
            & | \^{}    & P2 Binary ``or'', ``xor'' \\
            & >> <<     & P1 Binary shift left, right \\
equity       & > >=      & P3 Binary great, great or equal \\
            & < <=      & P3 Binary less, less or equal \\
            & == !=     & P3 Binary Equal, Not equal \\
logical     & \&\&      & P4 Binary logical ``and'' \\
            & ||        & P5 Binary logical ``or'' \\
assign      & = :=      & P6 Assign and assign and init \\
unary       & ++ --     & unary increment and decrement \\
            & -         & unary negative \\
            & \^{}      & unary bitwise complement \\
Pointer     & \& *      & unary get reference, dereference\\
priority    & ( )       & define the priorities in expression \\
scope       & \{ \}     & define the scope of variables and \\
punctuation & , ;       & \\
channels    & <-        & read, write value from chanel\\
slice       & \ldots    & extend the slice values \\
accessors   & .         & access of the property of structure \\
            & [ ]       & access of array/slice/map/string element \\
other       & : ~       &  \\
\end{tabular}

\end{leftcolumn} \begin{rightcolumn}
\textbf{Plus}, \textbf{minus} and \textbf{multiply} operations return
result of the same type as an operands.

\textbf{Devide} operator return float type.

\textbf{Less}, \textbf{great}, \textbf{equal}, \textbf{not eaqual}
operations return result of the boolean type.

Golang doesn't have a ternary operator whitch available in many
languages including
\end{rightcolumn}
\end{paracol}

\subsection*{File structure}
\begin{paracol}{2}
\begin{leftcolumn}
\begin{lstlisting}
package main

import "fmt"

// "main" function is a reserved as an entry poin into
/// the app and must be in "main" pacakge
func main() {
    fmt.Println("Hello world!")
}
\end{lstlisting}
\end{leftcolumn} \begin{rightcolumn}
Go file have a \textbf{package} name and could have \textbf{function},
\textbf{types}, \textbf{variables} or \textbf{constant} declaration.
\end{rightcolumn}
\end{paracol}

\newpage
\section*{Data types and structures}
\label{sec:data-types-and-structures}
GO is a strictly typed programming language.

\subsection*{Basic types}
\label{subsec:basic-types}
\begin{paracol}{2}
\begin{leftcolumn}
\begin{tabular}{llll}
\hline
Kind        & Default Value & Size    & Types       \\
\hline
Logical     & \texttt{false}& \texttt{1b}      & int8, uint8, byte, bool \\
Integers    & \texttt{0}    & \texttt{2b}      & int16, uint16 \\
            & \texttt{0}    & \texttt{4b}      & int32, uint32, rune\\
            & \texttt{0}    & \texttt{8b}      & int64, uint64\\
Numbers     & \texttt{0}    & \texttt{4b}      & float32 \\
            & \texttt{0}    & \texttt{8b}      & float64,  \\
            & \texttt{0}    & \texttt{8b}      & complex64  \\
            & \texttt{0}    & \texttt{16b}     & complex128 \\
Integers    & \texttt{0}    & \texttt{4b/8b}   & int, uint \\
Pointer     & \texttt{nil}  & \texttt{4b/8b}   & *int, *string, etc. \\
            & \texttt{0}    & \texttt{4b/8b}   & uintptr \\
String      & \texttt{""}   & \texttt{16b+}    & string \\
\hline
\end{tabular}
\end{leftcolumn}
\begin{rightcolumn}
1. Type casing by using T(value) \\
2. GO don't have a String interpolation feature. Use \texttt{Printf} and \\
\texttt{Sprintf} function from the standard library. \\
3. The standard library has flexible abilities to print value of any type with \\
\%v parameter.
\end{rightcolumn}
\end{paracol}

\subsection{Arrays}
Array is a fixed set of values of one data type.
\begin{lstlisting}
var a [2]string
a[0] = "Hello"
a[1] = "World"
fmt.Println(a[0], a[1])
fmt.Println(a)

primes := [6]int{2, 3, 5, 7, 11, 13}
fmt.Println(primes)
\end{lstlisting}

\subsection{Slices} have a dynamic size and don't store the own value.
Only represent the value of underlying array.

\begin{lstlisting}
primes := [6]int{2, 3, 5, 7, 11, 13}

var s []int = primes[1:4]
fmt.Println(s)
\end{lstlisting}

\texttt{len(s)} -- print length
\texttt{cap(s)} -- print capacity
\texttt{make([]int, 5)} -- create a slice
\texttt{append(s []T, vs \ldots T) []T}

Default value of slice is ``nil''

\textbf{Range} -- operator used to extract content of slice to individual values.

\begin{lstlisting}
var pow = []int{1, 2, 4, 8, 16, 32, 64, 128}
for i, v := range pow {
    fmt.Printf("2**%d = %d\n", i, v)
}
\end{lstlisting}

\subsection[maps]{Maps}
KAY-VALUE data structure with any type of key and any type of value.
\begin{lstlisting}
var myMap = map[KT]VT{
    key: value
}
... or ...
var myMap = make()
\end{lstlisting}

Delete

Ok? Statement.

\subsection{Structures}

\begin{lstlisting}
type Vertex struct {
    X int
    Y int
}
...
v := Vertex{1, 2}
\end{lstlisting}

\begin{Note} \textbf{Advanced note}
inner field promotion allows to use structures composition as an iheritance.
\end{Note}

Anonymous struct = inline struct

\newpage
\section{Code flow}
\label{sec:code-flow}

\subsection{Condition}
\begin{lstlisting}
if init; condition {
    ...
}
\end{lstlisting}

If else
\begin{lstlisting}
if init; condition {
    ...
} else {
    ...
}
\end{lstlisting}

Multiple select
\begin{lstlisting}
switch init; variable {
    case val1:
        ...
    case val2:
        ...
    default:
        ...
}
\end{lstlisting}

\begin{lstlisting}
switch {
    case condition1:
        ...
    case condition2:
        ...
    default:
        ...
}
\end{lstlisting}



\subsection{Loop}
\label{subsec:loop}
Go has only one loop which has init section, condition section and incremental section.
But only condition section is required

\begin{lstlisting}
for init; condition; increment{
    ...
}
\end{lstlisting}

Sample without init statement.
\begin{lstlisting}
for sum < 1000 {
    sum += sum
}
\end{lstlisting}

Or forever loop:
\begin{lstlisting}
for {
    ...
}
\end{lstlisting}

Range
\begin{lstlisting}
for key, value := range oldMap {
    newMap[key] = value
}
... or ...
for _, value := range oldMap {
    fmt.Println(value)
}
\end{lstlisting}

\textbf{Note, Advanced Topic}

GO don't have build-in control flow structure to iterate structure fileds.
Use `reflect` package for this.

\newpage
\section{Functions}

Declaration
\begin{lstlisting}
func (receiver) name (args) returnType {
...
}
\end{lstlisting}

\textbf{Defer} build the function execution stack
\begin{lstlisting}
func (args) type {
    defer fmt.Println("world")
    fmt.Println("hello")
}
\end{lstlisting}

Features

Multiple returns
Variadic parameters (Changed list of parameters)
\begin{lstlisting}
func Greet(names ...string) {
    name := "Guest"
    if len(names) > 0 {
        name = names[0]
    }
    fmt.Println("Hello", name)
}
\end{lstlisting}
Defer

\textbf{Receiver} is an ability to move function in the scope of some structure.
This is a way to simulate Object-Oriented approach

(!) Note, Advanced Topic (!)
No `default values` for functions parameters


\newpage
\section{Pointers}
\textbf{Pointer} is a variable that saves the reference to another variable.

\begin{paracol}{2} \begin{leftcolumn}
\begin{lstlisting}
a := 100
p := &a
fmt.Printf("Value of a: %d\n", a)
fmt.Printf("Address: %p, refered value is: %d", p, *p)
\end{lstlisting}
\end{leftcolumn} \begin{rightcolumn}
1. GO don't have a pointers arithmetics. +, - and * operation are not applicable
for Pointer type.
2. GO also doesn't have explicitly cast of pointers.
3. GO keep information about addressed size as part of Type Devinition at
compilation time.

\end{rightcolumn}
\end{paracol}

Memory conception.
\texttt{Var pInt *int} -- Declaration of pointer for integer.

I := 42
p = \&i -- get the pointer of the i
p       -- use the pointer
*p      -- use the pointer reverence value;
\section{Interface}

Type interface (Or Type Alias). More type aliases in
\href{constraint package}{https://pkg.go.dev/golang.org/x/exp/constraints}

\begin{lstlisting}
    type MyInt interface {
        int | uint
    }
\end{lstlisting}

\section{Generics}
\begin{paracol}{2} \begin{leftcolumn}
Generic function parameters.
\begin{lstlisting}
    func addT[T int | float64](a, b T) T {

    }
\end{lstlisting}

Constraints have a popular type aliases.

\end{leftcolumn}\begin{rightcolumn}
Generics in GO resolved at compilation time to not effect runtime performance
\end{rightcolumn}
\end{paracol}


\section{Error handler}
Part of buildin package

Panic / defer

Error interface

\section{Concurrency and goroutine}

\texttt{WaitGroup} -- include cimple synchronisatonn logic.

\texttt{Go} statement -- run function in gourutine

\texttt{Race Condition} -- is an issue when different pars of code race to one data.
Race Condition lead to data corruption

Select statement

\texttt{Mutex} the solution for accesing cucurrent resources.

\texttt{Atomic} similar to Mutext but works on processor level with simple operations.

Share by communication

\begin{lstlisting}
c1 := make(chan int) // create unbuffered channel
c2 := make(chan int, 8) // create buffered channel
c3 := make(chan <-int)  // create recevier-channel
c4 := make(<- chan int) // create send-channel
\end{lstlisting}

\texttt{make(\ldots)} used for creation of slices, maps and chanels.

operators \texttt{<-}send and recevie\texttt{<-} pause the goroutine until receiver-channel will have value to recive.
And pause the goroutine untile sender will not sure that value received.

Listening of chanel implemented by \textt{range} statement inside of \texttt{for} loop.
If chanel listened by somebody with range statement it must be closed notify all listeners about end of sendign.

If program have only sender (without recevier) of chanel it will stuck at sending. This is \textt{deadlock} situation.
The same issue if only receiver is waiting for a value from chanel, but the program doesn't have any sender.

Such program will crashed at Runtine with deadlock fatal error.

\texttt{close()} is build-in method used only to close the chanel.

\section{Modules and packages}

\textbf{Package} — is a logical entity and could be split to many files in one
directory.
Each directory can have only one (logical) package and name of
package and name directory could be different.

Each file could contain a lot of functions and types and other entities.


\textbf{AA, Module} — is a logical entity that represents the deliverable code.

\textbf{AA, Dependency} is a relationship between code of two and more modules.

\section{Development Infrastructure}
\label{sec:development-infrastructure}

\subsection{Go Compiler}

\texttt{Go doc} show the documentation and generate static documentation on
local machine

\subsection{Application}

Organization of command line application

Input / output stream

Error stream

Input parameters / exit code

\subsection{Operation System}

File System

Running Applications and Memory management.

\subsection{Unit testing}
\subsection{Documentation}

\texttt{> godoc -http: :8080} will start local server with all documentation at
http://localhost:8080 address

\subsection{Code Convention}

\section{Standard Library}

Package json.

\end{document}
