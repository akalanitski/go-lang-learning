%! Author = akalanitski
%! Date = 22.04.25

\documentclass[12pt]{article}
\usepackage[a4paper,left=2cm, right=1cm, top=1cm, bottom=2cm]{geometry}         % For setting margins

% Optional packages
 \usepackage[utf8]{inputenc}   % Encoding
\usepackage[T1]{fontenc}      % Better fonts
\usepackage{multicol}
\usepackage{listings}
\usepackage{xcolor}
\usepackage[framemethod=default]{mdframed}
\usepackage[scaled]{beramono}

% Make hyperlinks inside of document and clickable TOC
\usepackage[hidelinks]{hyperref}

% COLUMNS LAYOUT for notes
\usepackage{paracol}
\columnratio{0.7}
\setlength{\columnseprule}{0.4pt} % Line between columns
\setlength{\columnsep}{1em}       % Space between columns

\newmdenv[
    linecolor=gray,
    linewidth=1pt,
    roundcorner=5pt,
    backgroundcolor=gray!10,
    innertopmargin=1em,
    innerbottommargin=1em,
    innerleftmargin=1em,
    innerrightmargin=1em
]{Note}


% Go language definition for listings
\lstdefinelanguage{Go}{
    morekeywords={
        break, case, chan, const, continue, default, defer, else, fallthrough,
        for, func, go, goto, if, import, interface, map, package, range,
        return, select, struct, switch, type, var, nil, true, false, make,
        new, append, cap, copy, delete, len, close, complex, real, imag
    },
    sensitive=true,
    morecomment=[l]{//},
    morecomment=[s]{/*}{*/},
    morestring=[b]",
    morestring=[b]`,
}

% Style configuration
\lstset{
    language=Go,
    numbers=left,                % Line numbers on the left
    numberstyle=\tiny\color{gray},
    numbersep=14pt,
    basicstyle=\ttfamily\small,
    keywordstyle=\color{blue}\bfseries,
    commentstyle=\color{gray},
    stringstyle=\color{orange},
    showstringspaces=false,
    breaklines=true,
    tabsize=1,
}

\begin{document}

% First page with formal info and document structure.
% Title page
\title{Go Lang Notes}
\author{Aliaksei Kalanitski}
\date{\today}
\maketitle

My notes during basic couse of GO Lang couse. Cover only very basic topics like
Function, Variables, Intput, Output, Keywords

% Table of contents
\tableofcontents
\newpage


\section{Syntax}

\subsection{Keywords}
\begin{paracol}{2} \begin{leftcolumn}
\begin{tabular}{lllll}
break       & default       & func      & interface & \textbf{select} \\
case        & \textbf{defer} & \textbf{go} & \textbf{map} & struct \\
\textbf{chan} & else & \textbf{goto}    & package   & switch \\
const         & \textbf{fallthrough}    & if        & \textbf{range} & \textbf{type} \\
continue      & for         & import    & return    & var \\
\end{tabular}
\end{leftcolumn} \begin{rightcolumn}
According to \href{https://go.dev/ref/spec}{Go spec} it only 25 keywords to
write programs. I think it is not true. Standart data types are also reserved
and as well as some values (i.e., true, false, nil, NaN)
\end{rightcolumn}
\end{paracol}

\subsection{Identifiers}
\begin{tabular}{ll}
Blank identifier& \_ \\
Standart types  & any, comparable, error, uintptr and basic types \\
Constants       & true, false, iota \\
Values          & nil, NaN? \\
Functions       & append, cap, close, complex, copy, \\
                & delete, imag, len, make, new, panic, \\
                & print, real, recover

\end{tabular}
\subsection{Operators}
\begin{paracol}{2}
\begin{leftcolumn}
\begin{tabular}{lll}
math        & + -       & P2 binary plus and minus \\
            & * /       & P1 binary multiply, divide \\
            & \%        & P1 binary reminder \\
bitwise     & \& \&\^{} & P1 Binary "and", "not and" \\
            & | \^{}    & P2 Binary "or", "xor" \\
            & >> <<     & P1 Binary shift left, right \\
equty       & > >=      & P3 Binary great, great or equal \\
            & < <=      & P3 Binary less, less or equal \\
            & == !=     & P3 Binary Equal, Not equal \\
logical     & \&\&      & P4 Binary logical "and" \\
            & ||        & P5 Binary logical "or" \\
assign      & = :=      & P6 Assign and assigne and init \\
unary       & ++ --     & unary increment and decrement \\
            & -         & unary negative \\
            & \^{}      & unary bitwise complement \\
Pointer     & \& *      & unary get reference, dereference\\
priority    & ( )       & define the priorities in expression \\
scope       & \{ \}     & define the scope of variables and \\
punctuation & , ;       & \\
channels    & <-        & \\
slice       & \ldots    & extend the slice values \\
accessors   & .         & access of the property of structure \\
            & [ ]       & access of array,slice or map element \\
other       & : ~       &  \\
\end{tabular}

\end{leftcolumn} \begin{rightcolumn}
\begin{enumerate}
    \item\textbf{plus}, \textbf{minus} and \textbf{multiply} operations return
        result of the same type as an operands.
    \item \textbf{devide} operator return float type.
    \item \textbf{less}, \textbf{great}, \textbf{equal}, \textbf{not eaqual}
        operations return resul of boolean type.
    \item Golang doesn't have a ternary operator which available in many languages
        including
\end{enumerate}
\end{rightcolumn}
\end{paracol}

\subsection{File structure}
\begin{paracol}{2}
\begin{leftcolumn}
\begin{lstlisting}
package main

import "fmt"

// "main" function is a reserved as an entry poin into
/// the app and must be in "main" pacakge
func main() {
    fmt.Println("Hello world!")
}
\end{lstlisting}
\end{leftcolumn} \begin{rightcolumn}
Go file have a \textbf{package} name and could have \textbf{function},
\textbf{types}, \textbf{variales} or \textbf{constants} declaration.
\end{rightcolumn}
\end{paracol}

\newpage
\section{Data types and structures}
GO is a stricly typed programming language.
\subsection{Basic types}
\begin{paracol}{2}
\begin{leftcolumn}
\begin{tabular}{llll}
\hline
Kind        & Default Value & Size    & Types       \\
\hline
Logical     & \tt{false}& \tt{1b}      & int8, uint8, byte, bool \\
Integers    & \tt{0}    & \tt{2b}      & int16, uint16 \\
            & \tt{0}    & \tt{4b}      & int32, uint32, rune\\
            & \tt{0}    & \tt{8b}      & int64, uint64\\
Numbers     & \tt{0}    & \tt{4b}      & float32 \\
            & \tt{0}    & \tt{8b}      & float64,  \\
            & \tt{0}    & \tt{8b}      & complex64  \\
            & \tt{0}    & \tt{16b}     & complex128 \\
Integers    & \tt{0}    & \tt{4b/8b}   & int, uint \\
Pointer     & \tt{nil}  & \tt{4b/8b}   & *int, *string, etc. \\
String      & \tt{""}   & \tt{16b+}    & string \\
\hline
\end{tabular}
\end{leftcolumn}
\begin{rightcolumn}
1. Type casing by using T(value) \\
2. GO don't have a String interpolation feature. Use \tt{Printf} and \\
\tt{Sprintf} function from the standard library. \\
3. The standartd library have flexible abilities to print value of any type with \\
\%v paramter.
\end{rightcolumn}
\end{paracol}

\subsection{Arrays}
Array is a fixed side set of values of the same type.
var a [10]int
\begin{lstlisting}
var a [2]string
a[0] = "Hello"
a[1] = "World"
fmt.Println(a[0], a[1])
fmt.Println(a)

primes := [6]int{2, 3, 5, 7, 11, 13}
fmt.Println(primes)
\end{lstlisting}


\subsection{Slices}
Slices has a dynamic size and don't store the own value. Only represent the value of underlying array.
\begin{lstlisting}
primes := [6]int{2, 3, 5, 7, 11, 13}

var s []int = primes[1:4]
fmt.Println(s)
\end{lstlisting}

len(s)  -- print length
cap(s) -- print capacity
make([]int, 5)
append(s []T, vs ...T) []T

default value of slice is "nil"

\b{Range} -- operator used to extract content of slice to individual values.
\begin{lstlisting}
var pow = []int{1, 2, 4, 8, 16, 32, 64, 128}
for i, v := range pow {
    fmt.Printf("2**%d = %d\n", i, v)
}
\end{lstlisting}

\subsection[maps]{Maps}
KAY-VALUE data structure with any type of key and any type of value.
\begin{lstlisting}
var myMap = map[KT]VT{
    key: value
}
... or ...
var myMap = make()
\end{lstlisting}

delete

ok? statement.

\subsection{Structures}

\begin{lstlisting}
type Vertex struct {
    X int
    Y int
}
...
v := Vertex{1, 2}
\end{lstlisting}

\begin{Note} \textbf{Advanced note}
inner field promotion allow to use structures composition as an iheritance.
\end{Note}

Anonymous struct = inline struct

\newpage
\section {Code flow}

\subsection{Condition}
\begin{lstlisting}
if init; condition {
    ...
}
\end{lstlisting}

if ... else

\begin{lstlisting}
if init; condition {
    ...
} else {
    ...
}
\end{lstlisting}

Multiple select
\begin{lstlisting}
switch init; variable {
    case val1:
        ...
    case val2:
        ...
    default:
        ...
}
\end{lstlisting}

\begin{lstlisting}
switch {
    case condition1:
        ...
    case condition2:
        ...
    default:
        ...
}
\end{lstlisting}



\subsection{Loop}
Go has only one loop which has init section, condition section
and incremental section. But only condition section is required
\begin{lstlisting}
for init; condition; increment{
    ...
}
\end{lstlisting}

Sample without init statement.
\begin{lstlisting}
for sum < 1000 {
    sum += sum
}
\end{lstlisting}

Or forever loop:
\begin{lstlisting}
for {
    ...
}
\end{lstlisting}

range
\begin{lstlisting}
for key, value := range oldMap {
    newMap[key] = value
}
... or ...
for _, value := range oldMap {
    fmt.Println(value)
}
\end{lstlisting}

\textbf{Note, Advanced Topic}

GO don't have build-in control flow structure to iterate structure fileds. Use
`reflect` package for this.

\newpage
\section{Functions}

Declaration
\begin{lstlisting}
func (receiver) name (args) returnType {
...
}
\end{lstlisting}

\textbf{defer} build the function execution stack
\begin{lstlisting}
func (args) type {
    defer fmt.Println("world")
    fmt.Println("hello")
}
\end{lstlisting}

Features

Multiple returns
Variadic parameters (Changed list of parameters)
\begin{lstlisting}
func Greet(names ...string) {
    name := "Guest"
    if len(names) > 0 {
        name = names[0]
    }
    fmt.Println("Hello", name)
}
\end{lstlisting}
Defer

\textbf{Receiver} is an ability to move function in scope of some structure. This is
a way to simulate Object-Oriented approach

(!) Note, Advanced Topic (!)
No `default values` for functions parameters


\newpage
\section{Pointers}
\textbf{Pointer} is a variable which save the reference to other variable.

\begin{paracol}{2} \begin{leftcolumn}
\begin{lstlisting}
a := 100
p := &a
fmt.Printf("Value of a: %d\n", a)
fmt.Printf("Address: %p, refered value is: %d", p, *p)
\end{lstlisting}
\end{leftcolumn} \begin{rightcolumn}
1. GO don't have a pointers arythmetics. +, - and * operation are not applicable
for Pointer type.
2. GO also doesn't have explicitly cast of pointers.
3. GO keep information about addressed size as part of Type Devinition at
compilation time.

\end{rightcolumn}
\end{paracol}

Memory conception.
\texttt{var pInt *int} -- Declaration of pointer for integer.

i := 42
p = \&i -- get the poiner of the i
p       -- use the pointer
*p      -- use the pointer reverence value;
\section{Interface}

Type interface (Or Type Alias). More type aliases in
\href{constraint package}{https://pkg.go.dev/golang.org/x/exp/constraints}

\begin{lstlisting}
    type MyInt interface {
        int | uint
    }
\end{lstlisting}

\section{Generics}
\begin{paracol}{2} \begin{leftcolumn}
Generic function parameters.
\begin{lstlisting}
    func addT[T int | float64](a, b T) T {

    }
\end{lstlisting}

Constraints have a popular type aliases.

\end{leftcolumn}\begin{rightcolumn}
Generics in GO resolved at compilation time to not effect runtime performance
\end{rightcolumn}
\end{paracol}


\section{Error handler}
Part of buildin package in


\section{Concurrency}

WaitGroup

go statement

Race Condition problem

Select statement

Mutex

Atomic


\section{Modules and packages}

\textbf{Package} is a logical entity and could be splited to many files in one
directory.


Each file could contains a lot of functions and types and other entities.


\textbf{AA, Moduel} is logical entity which represent the deliverable code.


\textbf{AA, Dependency} is a relationship between code of two and more modules.

\section{Development Infrastructure}

\subsection{Go Compiler}

\tt{> go doc} show the documentation and generate static documentation on local
machine

\subsection{Unit testing}
\subsection{Documentation}
\subsection{Code Convention}

\section{Standart Library}

package json.

\end{document}
